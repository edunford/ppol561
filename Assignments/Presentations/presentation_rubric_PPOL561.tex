% Options for packages loaded elsewhere
\PassOptionsToPackage{unicode}{hyperref}
\PassOptionsToPackage{hyphens}{url}
%
\documentclass[
]{article}
\usepackage{lmodern}
\usepackage{amssymb,amsmath}
\usepackage{ifxetex,ifluatex}
\ifnum 0\ifxetex 1\fi\ifluatex 1\fi=0 % if pdftex
  \usepackage[T1]{fontenc}
  \usepackage[utf8]{inputenc}
  \usepackage{textcomp} % provide euro and other symbols
\else % if luatex or xetex
  \usepackage{unicode-math}
  \defaultfontfeatures{Scale=MatchLowercase}
  \defaultfontfeatures[\rmfamily]{Ligatures=TeX,Scale=1}
\fi
% Use upquote if available, for straight quotes in verbatim environments
\IfFileExists{upquote.sty}{\usepackage{upquote}}{}
\IfFileExists{microtype.sty}{% use microtype if available
  \usepackage[]{microtype}
  \UseMicrotypeSet[protrusion]{basicmath} % disable protrusion for tt fonts
}{}
\makeatletter
\@ifundefined{KOMAClassName}{% if non-KOMA class
  \IfFileExists{parskip.sty}{%
    \usepackage{parskip}
  }{% else
    \setlength{\parindent}{0pt}
    \setlength{\parskip}{6pt plus 2pt minus 1pt}}
}{% if KOMA class
  \KOMAoptions{parskip=half}}
\makeatother
\usepackage{xcolor}
\IfFileExists{xurl.sty}{\usepackage{xurl}}{} % add URL line breaks if available
\IfFileExists{bookmark.sty}{\usepackage{bookmark}}{\usepackage{hyperref}}
\hypersetup{
  hidelinks,
  pdfcreator={LaTeX via pandoc}}
\urlstyle{same} % disable monospaced font for URLs
\usepackage[margin=1in]{geometry}
\usepackage{graphicx,grffile}
\makeatletter
\def\maxwidth{\ifdim\Gin@nat@width>\linewidth\linewidth\else\Gin@nat@width\fi}
\def\maxheight{\ifdim\Gin@nat@height>\textheight\textheight\else\Gin@nat@height\fi}
\makeatother
% Scale images if necessary, so that they will not overflow the page
% margins by default, and it is still possible to overwrite the defaults
% using explicit options in \includegraphics[width, height, ...]{}
\setkeys{Gin}{width=\maxwidth,height=\maxheight,keepaspectratio}
% Set default figure placement to htbp
\makeatletter
\def\fps@figure{htbp}
\makeatother
\setlength{\emergencystretch}{3em} % prevent overfull lines
\providecommand{\tightlist}{%
  \setlength{\itemsep}{0pt}\setlength{\parskip}{0pt}}
\setcounter{secnumdepth}{-\maxdimen} % remove section numbering

\title{PPOL561 \textbar{} Accelerated Statistics for Public Policy II\\
Presentation Rubric}
\author{}
\date{\vspace{-2.5em}}

\begin{document}
\maketitle

\hypertarget{presenters-_____________________________}{%
\section{Presenter(s)
\_\_\_\_\_\_\_\_\_\_\_\_\_\_\_\_\_\_\_\_\_\_\_\_\_\_\_\_\_}\label{presenters-_____________________________}}

\hypertarget{date-________-week-_____}{%
\section{Date \_\_\_\_\_\_\_\_ (Week: \_\_\_\_\_
)}\label{date-________-week-_____}}

\hypertarget{overview}{%
\section{Overview}\label{overview}}

The presentation is a 10 minute in-class presentation with slides on a
paper related to the material we are discussing. These presentations
will be done in teams of two. Each team will be responsible for locating
a research paper published in a peer-reviewed journal. The presentation
should summarize the substantive and statistical issues addressed in the
paper and provide context and a critique.

The following rubrics outlines how presenters will be evaluated when
presenting in PPOL561. A total of 50 points are available across 5
categories: preparedness, presentation performance, slides, critique,
and timing. Grades will not be assigned to student until \emph{all}
student groups have presented to ensure that grades appropriately
reflect the performance distribution of the entire class. Students will
be graded individually although there will likely be high correlation
between grades of their partner.

\hypertarget{rubric}{%
\section{Rubric}\label{rubric}}

\hypertarget{total-points-____50}{%
\subsection{Total Points: \_\_\_\_/50}\label{total-points-____50}}

\hypertarget{prepared-points-___10}{%
\subsection{Prepared (Points \_\_\_/10)}\label{prepared-points-___10}}

\begin{itemize}
\tightlist
\item
  \textbf{\emph{Criteria}}

  \begin{itemize}
  \tightlist
  \item
    \textbf{Unsatisfactory} (0-3 points): \emph{Little time practicing
    and preparing presentation; no evidence that the team practiced
    together; did not send slides to instructor the day before.}
  \item
    \textbf{Acceptable} (4-7 points): \emph{Spent time practicing and
    preparing presentation but parts were choppy; some evidence that the
    team practiced together; sent slides to instructor the day before.}
  \item
    \textbf{Excellent} (8-10 points): \emph{Clearly spent time
    practicing and preparing presentation; strong evidence that the team
    practiced together; sent slides to instructor the day before.}
  \end{itemize}
\item
  \textbf{\emph{Notes}}:
\end{itemize}

\begin{verbatim}
  
  
  
  
\end{verbatim}

\pagebreak

\hypertarget{presentation-points-___10}{%
\subsection{Presentation (Points
\_\_\_/10)}\label{presentation-points-___10}}

\begin{itemize}
\tightlist
\item
  \textbf{\emph{Criteria}}

  \begin{itemize}
  \tightlist
  \item
    \textbf{Unsatisfactory} (0-3 points): \emph{The team members did not
    present well together (stumbling often when transitioning between
    speakers); concepts/logic were difficult to follow; presenters were
    difficult to understand. Presenters crammed too much (or not enough)
    material into the presentation; unclear why the presenters focused
    on what they presented. Presenters read off of their slides.}
  \item
    \textbf{Acceptable} (4-7 points): \emph{The team members presented
    well enough together (stumbling at times when transitioning between
    speakers); concepts were at times difficult to follow; presenters
    enunciated most of the time and generally spoke in manner that
    others could understand, but at times they were difficult to follow.
    Presenters did crammed a lot of material into the presentation
    (tried to cover most of the paper and/or focused on specific aspect
    of the paper but it was unclear why). Presenters sometimes read off
    of their slides.}
  \item
    \textbf{Excellent} (8-10 points): \emph{The team members presented
    well together (rarely if ever stumbling when transitioning between
    speakers); Presentation was well crafted to the audience; concepts
    were easy to follow; presenters enunciated well and spoke in manner
    that others could understand and follow. Presenters did not cram too
    much material into the presentation (clearly focused on specific
    aspect of the paper and made this known). Presenters never read off
    of their slides.}
  \end{itemize}
\item
  \textbf{\emph{Notes}}:
\end{itemize}

\begin{verbatim}
  
  
  
  
  
  
  
  
  
  
\end{verbatim}

\hypertarget{slides-points-___10}{%
\subsection{Slides (Points \_\_\_/10)}\label{slides-points-___10}}

\begin{itemize}
\tightlist
\item
  \textbf{\emph{Criteria}}

  \begin{itemize}
  \tightlist
  \item
    \textbf{Unsatisfactory} (0-3 points): \emph{Minimal slides; slides
    appear rushed; little thought was put into visual presentation.}
  \item
    \textbf{Acceptable} (4-7 points): \emph{Good slides; a bit wordy at
    times; used graphics but they were not visually appealing; relied on
    tables at parts.}
  \item
    \textbf{Excellent} (8-10 points): \emph{Clean and clear slides;
    limited number of words on each slide; compelling graphics; little
    or no tables; slides transitioned well.}
  \end{itemize}
\item
  \textbf{\emph{Notes}}:
\end{itemize}

\begin{verbatim}
  
  
  
  
\end{verbatim}

\pagebreak

\hypertarget{critique-points-___10}{%
\subsection{Critique (Points \_\_\_/10)}\label{critique-points-___10}}

\begin{itemize}
\tightlist
\item
  \textbf{\emph{Criteria}}

  \begin{itemize}
  \tightlist
  \item
    \textbf{Unsatisfactory} (0-3 points): \emph{Offered a poor critique
    of the paper (if any); did not include an explanation for why their
    critique would alter the results/conclusions made in the paper (or
    their explanation was poorly formulated)}
  \item
    \textbf{Acceptable} (4-7 points): \emph{Offered an satisfactory
    critique of the paper; included an okay explanation for why their
    critique would alter the results/conclusions made in the paper.}
  \item
    \textbf{Excellent} (8-10 points): \emph{Well formulated critique of
    the paper; included a compelling explanation for why their critique
    would alter the results/conclusions made in the paper.}
  \end{itemize}
\item
  \textbf{\emph{Notes}}:
\end{itemize}

\begin{verbatim}
  
  
  
  
  
  
  
  
  
  
  
\end{verbatim}

\hypertarget{timing-points-___10}{%
\subsection{Timing (Points \_\_\_/10)}\label{timing-points-___10}}

\begin{itemize}
\tightlist
\item
  \textbf{\emph{Criteria}}

  \begin{itemize}
  \tightlist
  \item
    \textbf{Unsatisfactory} (0-3 points): \emph{Presentation went 2+/-
    minutes over/under the required 10 minutes (instructor had to stop
    them). Presenters rushed through material.}
  \item
    \textbf{Acceptable} (4-7 points): \emph{Presentation went 1+/-
    minute over/under the required 10 minutes. Good flow; presenters
    rushed through some parts of the material.}
  \item
    \textbf{Excellent} (8-10 points): \emph{Presentation took exactly 10
    minutes. Well executed; presenters did not appear to rush.}
  \end{itemize}
\item
  \textbf{\emph{Notes}}:
\end{itemize}

\begin{verbatim}
  
  
  
  
  
  
\end{verbatim}

\end{document}
